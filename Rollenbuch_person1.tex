\documentclass[12pt]{article}
%\documentclass[pdf,final,12pt,a5paper,twoside]{article}

%\special{papersize=148mm,210mm}
\usepackage[a5paper,twoside,bindingoffset=1.5cm,inner=1.2cm,outer=1.5cm,bottom=8cm,top=-3cm,headsep=1.0cm]{geometry}
\usepackage[sqrt2]{scale}
\usepackage[ngerman]{babel}
\usepackage[latin1]{inputenc}

\usepackage{fancyhdr}
%%%%%%%%%%%%%%%%%%%%%%%%%%%%%%%%%%%%%%%%%%%%%%%

\pagestyle{fancy}

\fancyhead{}
\fancyhead[LE,RO]{\thepage}
\fancyhead[RE]{Rollenbuch person1}
\fancyhead[LO]{\leftmark}
\fancyfoot[CE,CO]{}
\fancypagestyle{plain}{
\fancyhead{}
\fancyhead[LE,RO]{\thepage}
\fancyhead[LO]{\rightmark}
\fancyhead[RE]{\leftmark}
\fancyfoot[CE,CO]{}
}
\fancypagestyle{realplain}{
\fancyhead{}
\fancyfoot[CE,CO]{\thepage}
}
\setlength{\headheight}{15pt}
%\addtolength{\headwidth}{3cm}
%\addtolength{\textwidth}{1.5cm}
%\addtolength{\evensidemargin}{-1.5cm} 
%\setlength{\oddsidemargin}{1cm}

\setcounter{secnumdepth}{-2}


\newcommand{\Stop}{\fbox{\parbox{\textwidth}{\begin{center}\bf \Large STOPP:\\ Nicht weiterlesen, bis du dazu aufgefordert wirst!\end{center}}} \newpage}
\begin{document}

\begin{center}
\LARGE
Rollenbuch\\ \vspace{1cm}
person1\\ \vspace{1cm}
Titel
\normalsize
\end{center}
\newpage

\pagestyle{fancy}
\section{Spielregeln}
\begin{enumerate}
\item Es geht in diesem Spiel um Leben, Mord und Tod. Lass dir davon den Abend nicht vermiesen, sondern genie�e es.
\item Jeder von Euch mag an irgendeinem Punkt verd�chtig sein. Du sch�tzt dich, wenn du Fragen stellst und den wahren M�rder findest.
\item Zeige dein Rollenbuch niemandem und lese nicht weiter als bis zur n�chsten STOPP Marke, bevor du dazu aufgefordert wirst.
\item Du darfst nicht l�gen, wenn du direkt auf etwas angesprochen wirst, aber Ablenkung, Gegenangriff, Ausweichen, ... sind erlaubt.
\item Um die Hinweise zu strukturieren, ist das Spiel in drei Runden eingeteilt. Jede Runde beginnt mit einem Gespr�ch, das laut vorgelesen wird. Dies dient dazu, Hinweise zu geben und die Positionen der Charaktere darzustellen.
\item Die geheimen Hinweise, die du bekommst, d�rfen erst in der angegebenen Runde ge�ffnet werden.
\item Nach Runde 3 gibt es noch einmal eine Zusammenfassung. Dann soll jeder den T�ter und den Tathergang erraten. 

\end{enumerate}

\newpage

\section{Einleitung}
Wir schreiben das Jahr 1973. Die Watergate Aff�re um Pr�si\-dent Nixon hat zu den ersten R�cktritten gef�hrt. Die �f\-fent\-lich\-keit ist schockiert. Gleichzeitig hat Elvis Presley mit seinem Konzert \glqq Aloha from Hawaii\grqq{} weltweit 1 Million Menschen vor den Fernsehbildschirmen begeistert. 
Auch in Deutschland herrschen Latzhosen und Kordletten, w�harend die RAF Angst und Schrecken verbreitet. Politishc wirft der unberechenbare russische B�r seine Schatten �ber Deutschland, und gleichzeitig versetzt die �lkrise der Wirtschaft einen schweren Schlag. An den Universit�ten aber ist 5 Jahre nach den studentischen Unruhen der 68iger wieder Ruhe eingekehrt. Wirklich?\\

Eure kleine Gruppe von Chemiestudenten hat sich hier im Besprechungsraum von Professor Eisenmaier versammelt, damit er euch die Noten in den theoretischen Klausuren mitteilen kann. Im Anschluss will er noch das Vorgehen in der kommenden praktischen Pr�fung mit euch besprechen. Ihr habt gerade eure Pl�tze eingenommen, als Kommissar Steiner hereinplatzt.


\section{Verd�chtige}

\subsection{Ingolf Kleinen}
Dieser Text ist f�r die �ffentliche Personen beschreibung und k�nnte so anfangen:
Ingolf Kleinen (25) verk�rpert den typischen Studenten wie kaum eine anderer --- weiteren Text einf�gen. 

%\subsection{Hartmut Fuchs}
%\input{Hartmut_allg}



\section{Das bist Du}
\subsection{Was alle sehen}
Dieser Text ist f�r die �ffentliche Personen beschreibung und k�nnte so anfangen:
Ingolf Kleinen (25) verk�rpert den typischen Studenten wie kaum eine anderer --- weiteren Text einf�gen. 

\subsection{Was nur du wei�t}
Dieser Text ist f�r die private Einleitung. Er beschreibt den nicht-�ffentlichen Hintergrund einer Person. Einge diese Fakten werden im Laufe des Spiels herauskommen. Der Text k�nnte so beginnen:

Dein Gro�vater ist im KZ von einem Chemiker f�r pharmakologische Experimente missbraucht ...

\vspace{3mm} \\ \Stop


\section{1. Runde}
\input{R1_Gespraech}\newpage
\subsection{Hinweise f�r diese Runde:}
\begin{itemize}\input{person1_R1}\end{itemize}
\Stop

\section{2. Runde}
\input{R2_Gespraech}\newpage
\subsection{Hinweise f�r diese Runde:}
\begin{itemize}\item Fakt um in dieser Runde jemanden anzuklagen

\item anderes Fakt um in dieser Runde jemanden anzuklagen
\item Vielleicht auch eine Fakt, um dich zu verteidigen.
\end{itemize}
\Stop

\section{3. Runde}
\begin{description}
\item[Kommissar]
Meine Damen und Herren, Herr Professor Eisenmaier ist bei einem Experiment in der Erstsemestervorlesung explodiert ... 

\item[Person 1]
Mmmhhh, zwei Wochen nach Semesterbeginn, da hat er bei uns die verschiedenen Zust�nde von Phosphor gemacht.

\end{description}
\newpage
\subsection{Hinweise f�r diese Runde:}
\begin{itemize}\item extra fact to accuse somebody

\item more extra facts
\item and maybe some fact / hints to fence of accusations
\end{itemize}
\Stop

\section{Zusammenfassung}
% Zusammenfassung der Erkenntnisse bevor die Spielter raten.

\begin{description}
\item[Kommissar]
So meine Damen und Herren,
hier jetzt unser Ermittlungsstand:
mit ihrer Hilfe konnten wir den Fall voll\-st�n\-dig l�sen. Wenn man einen Mord aufkl�ren will, dann sucht man immer zuerst nach dem Motiv. Und da sind wir bei ihnen allen f�ndig geworden. Sie haben es ja selbst gesagt. 
Damit sich keiner bevorzugt f�hlt, gehen wir mal nach dem Alphabet.

person1, sie waren ja zur Tatzeit...

\end{description}

\vspace{1.5cm}
Nun sind hoffentlich alle Tatsachen bekannt. Jeder Spieler muss jetzt sein Votum abgeben, wer der T�ter war, indem er klar und deutlich einen Namen sagt oder auf einem Zettel notiert.
Erst wenn alle Spieler sich festgelegt haben, darf eine Diskussion �ber den Tatablauf, die Motive oder anderes beginnen.\\
\Stop

\section{Aufl�sung}
Hier kommt die Aufl�sung hin. Ca. 1-2 Seiten Text sind durchaus angemessen f�r einen komplizierten Tathergang.


\vspace{1.5cm}
Hinweise und Verbesserungsvorschl�ge bitte an uns:\\
\texttt{moritz.guenther@gmx.de \\ silkeopitz@web.de}\\
Version: \today
\end{document}
